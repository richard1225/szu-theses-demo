\section{引言}

\subsection{研究背景及意义}

互联网的出现和普及给用户带来了大量的信息, 满足了用户在信息时代对各种信息的需求, 但随着Internet的迅速发展而带来的网络上信息量的巨幅增长, 使得用户在面对大量信息时无法快速从中获得对自己真正有用的那部分信息。换言之, 在这种情况下人们对信息的使用效率反而降低了, 这就是所谓的信息过载(information overload)问题. 的确如此, 面对信息的汪洋大海, 人们往往感到无所适从, 信息过载已经成为一个不容忽视的问题.

目前, 应对信息过载的办法之一便是以搜索引擎为代表的信息检索系统, 比如国外的Google\footnote{\url{https://www.google.com/}}、国内的Baidu\footnote{\url{https://www.baidu.com/}}等, 它们在帮助用户从巨大的网络资源中获取信息方面发挥着极其重要的作用. 但对于使用搜索引擎的用户而言, 在使用同一个关键字搜索信息时, 在一段时间内所得到的结果都是相同的. 另一方面来看,信息及其传播是多样化的, 而用户对信息的需求是多元化和个性化的, 那么通过以搜索引擎为代表的信息检索系统获得的结果显然不能满足用户的个性化需求, 它们仍然无法很好地解决信息过载问题.

面对信息过载, 另外一个非常有潜力的办法是个性化的推荐系统, 它是根据用户的信息需求、兴趣等, 将用户所感兴趣的信息、产品、服务等推荐给用户的个性化信息推荐系统. 和搜索引擎相比, 推荐系统通过研究用户的历史行为与兴趣偏好, 进行个性化考量, 由系统发现用户的兴趣点, 从而引导用户发现自己的信息需求. 一个优秀的推荐系统不仅能为用户提供个性化的服务, 还能和用户之间建立密切关系, 让用户对其推荐产生依赖. 个性化推荐系统现已广泛应用于很多领域, 其中最典型并具有良好的发展和应用前景的领域就是电子商务领域. 目前,几乎所有大型的电子商务系统,如 Amazon, eBay, 京东, 当当网上书店等, 都不同程度地使用了各种形式的推荐系统。同时学术界对推荐系统的研究热度一直很高, 逐步形成了一个独立的研究领域.

Internet为人们提供了极其丰富的信息资源,在这些海量、异构的Web信息资源中蕴含着具有巨大潜在价值的知识。根据用户访问的历史记录以及各种服务或商品之间的相关信息可以构建用户的兴趣模型,从而凭借该用户的兴趣模型对繁杂的信息进行过滤, 然后向用户推荐其可能感兴趣的服务或商品。事实上, 推荐系统已经成为目前解决信息过载最有效的工具之一。




\subsection{本文主要工作}

本文从推荐系统的概述展开, 讨论了在推荐系统的学习算法中随机梯度下降方式中采用均匀采样策略而导致收敛缓慢的一些原因, 并通过融合内容信息改进了均匀采样策略--适应性采样策略, 然后将适应性采样策略放入已有的推荐算法框架中, 加快原有推荐算法的学习。




\subsection{论文组织结构}

 本论文共分为七章,内容如下: 

 第一章为引言, 主要介绍了本论文的研究背景、意义, 主要工作及论文的组织结构.
 
 第二章为推荐系统概览,并分类介绍了包括了基于内容、基于系统过滤与混合型推荐算法的一些典型的推荐学习算法。
 
 第三章为预备工作,首先简要回顾了Bayesian Personalized Ranking(BPR)推荐算法, 并对其局限性进行了一些探讨。
 
 第四章为适应性采样策略,主要研究了通过融合内容信息提出了适应性采样策略改进已有的均匀采样策略。
 
 第五章为整体的算法框架, 将适应性采样策略融入已有的BPR推荐模型。
 
 第六章为实验论证,主要内容为在适应性采样策略下的推荐算法的实验表现。
 
 第七章为结论与展望,首先简要总结了本文的一些工作,并对接下来进一步的研究工作做了展望。
