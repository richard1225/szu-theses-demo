\section{算法实现与结果分析}
\label{analize}

第\ref{analize}章将对以上提到的算法和改进进行实证研究,以评估算法的效率和改进方案。\ref{analize_subsection_1}节将介绍实验中使用的数据集、查询参数以及实验平台。4.2节将评估基于密度k-STC查询的改进方案。

\subsection{实验环境}
\label{analize_subsection_1}

本章的数据集来自网站kaggle上188593组餐厅信息,每组信息包括店铺ID,地理位置以及菜单信息。每家餐厅的文字描述长度平均为8个单词。具体来说,我们将18万组数据生成了大小为10k,50k,100k,150k,180k的数据集。利用不同的数据集测试算法性能。在180k的数据集中

在180k的数据集中随机选择一个对象p,利用位置偏离得到一个新的对象,新的对象文本描述保持一致。从对象的文本中随机选取单词作为查询关键字,关键词数量为1,2,3。因为查询集是对象文本的子集,所以不会有查询为空的情况。

实验评估了基本方法(basic)、带权重的IR-树的优化方法(Adv)。表4-1显示了实验中使用的参数值,其中粗体值是默认值。所有算法均采用Java语言实现,实验采用Intel(R) Core(TM) i5-4590 CPU @ 3.30GHz,内存8GB。所有的数据结构都保存在内存中。实验记录k-STC查询所需的平均运行时间。

\begin{document}
\begin{center}
\renewcommand\arraystretch{1.3}
\begin{tabular}{|c|c|c|c|c|}
 \hline
 Text A     & Text B    & Text C    & Text D    & Text E    \\
 \hline
 A          & Text F    & Text G    & Text H    & Text I    \\
 \hline
 B          & Text J    & Text K    & Text L    & Text M    \\
 \hline
 \multirow{3}*{Text N}
            & \multirow{3}*{Text O}
                        & Text P    & Text Q    & Text R    \\
 \cline{3-5}
            &           & \multirow{2}*{Text S}
                                    & Text T    & Text U    \\
 \cline{4-5}
            &           &           & Text V    & Text Z    \\
\hline
\end{tabular}
    \end{center}
\end{document}