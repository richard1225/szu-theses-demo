\section{引言}

\subsection{研究背景及意义}
随着社会的进步与发展,以及信息逐渐趋向透明、公开化,越来越多案件的审判过程与结果暴露在公众的视野下,接受着公众的监督。然而,在同一件案子下,不同的法官可能会得出不同的审判结果,这是因为每个人都有自己的评判标准。在此情形下,让案件得到公正的审判、减少个人尺度标准差异带来的影响,变得尤为重要。通过本论文提出的方法,能够训练出一个模型,该模型可以从当前判决书中提取出关键因子,再通过多项关键因子推荐相似的已判决过的案件,这些案件给法官作为一个参考,以减少个人因素带来的影响,除此之外,此技术还能帮助减轻法官,律师的工作量,让他们能把精力更多地投入到案子。这能带来一个更公正的审判结果,从而引导一个和谐的社会舆论氛围。

随着最高人民法院对审判流程信息、裁判文书信息、执行信息全面公开的推进,中国裁判文书网在此背景下诞生,作为全国法院统一的裁判文书公开平台,裁判文书生效后七日内都将被传送至中国裁判文书网公布累计只2017年的统计,库内判决书已达3247万篇\cite{WEB:judgement},这繁杂的信息中,有很多信息对律师法官,乃至人民群众来说,都有很高的参考价值,如何利用好这么大的数据,就要使用到自然语言处理技术(NLP)

律师与法官在接受到一个案子的时候,首先要做的就是从案件中提取关键点。关键点就是这个案子的争议焦点、历史判决、双方诉称辩称、引用的法律发条等等。然后,根据这些关键点,去裁判文书网中搜索类似的案件作为参考。但是这样检索存在两个问题,一是关键点通常都是一句陈述句,并且在不同案件下,同一个意思往往有好几种表达方法,这类关键句在裁判文书网中往往检索不到想要的结果。二是检索到的结果都是弱相关的,仍然需要用户进一步自行筛选。这样的流程得出来的结果是低效,精准度差的。

得益于数据挖掘中的文本挖掘的发展,以上痛点均可通过自然语言处理技术来解决。首先从裁判文书网获取若干判决书作为训练姐,再对训练集进行中文预处理,以句来切分,对所有的句子使用Kmeans聚类算法来聚类,从聚类结果中挑选出符合关键因子特征的簇,作为一类关键因子。通过反复的迭代与挑选,可以选出足够数量的簇,足以覆盖绝大多数的关键点。再用这些簇来训练出一个分类模型,即可得出一个能从新的判决书中分类出关键句子的模型。在这个过程中,聚类的质量直接影响了分类模型的精准度。本人在此项目中主要从事聚类相关工作,因此本文主是要围绕聚类算法来展开研究,对比多种聚类算法的质量。

\subsection{国内外研究现状}
虽然很早之前,就有人开始研究自然语言处理,但真正提出用计算机来处理自然语言,是在1950年。当时,艾伦图灵提出了一个标准,意为能通过”图灵测试“\cite{WEB:turing_test}标准的程序可以被判断为是智能的。概念提出之后一直到1980年代,人们研发的NLP系统都是基于规则的,由复杂的规则堆砌而成,那时候的人们调侃这种系统为“积木系统”,对于超过规则之外的输入,这种系统只能给出机械式的,无关痛痒的回答。1980年之后,人们把机器学习引入了NLP来提升它的准确性和稳定性。除此之,NLP还因为两个革新效果大幅度提升,一是运算能力稳定增加(硬件的性能和价格遵守摩尔定律),二是转换-生成文法方法的乔姆斯基语言学理论不再是NLP的主要方法\cite{WEB:turing_test}。从此之后,各领域的NLP技术稳步发展,逐渐走向成熟。

近期,已经有人开始把深度学习的技巧也引入NLP,在自然语言处理方面取得了巨大的成就,如Yonghui Wu等人发表的Exploring the Limits of Language Modeling\cite{DBLP:journals/corr/JozefowiczVSSW16}。但本文只研究基于无监督学习的聚类算法。

作为NLP的重要分析手段,聚类分析,是当下的一个热门研究对象。在聚类分析的发展进程中,产生了大量的聚类算法,主流的为以下几种:中心聚类、层次聚类、密度聚类、谱聚类等。

聚类算法的目的,通俗地来讲,是将具有共性的样本归类为统一类别,再将少有甚至没有共性的样本归类为不同类别。这里的共性通常是通过度量样本之间的距离来求得的,对距离的定义有很多种,我们需要结合样本的特点来具体分析。\cite{ZW:cluster_alg_study_compare}

总之,聚类算法最终是要得到一批符合以下亮点特征的簇的集合:一是每个簇(又称为类簇)内的距离相对紧密,二是不同簇的簇间距离相对较大。

\subsection{研究目的与主要工作}
本文主要研究的聚类算法是KMeans\cite{Macqueen67somemethods}算法和GSDMM\cite{Yin:2014:DMM:2623330.2623715}算法

\subsubsection{KMeans算法}

KMeans算法的主要思想是寻找每个簇的中心,再把与中心具有共性的点归类的该中心所代表的类簇中。KMeans算法是迭代执行的,需要给出有一个参数K,K为预估的类簇的数量。初始的时候,会随机生产K个中心点,在KMeans的每轮迭代中,都会更新K个中心点的位置,重复迭代,直到K个中心点的位置不再变化,此时可以判断为收敛了,所得的K个点的集合,经过筛选之后,即可选为最终的类簇。用KMeans算法的好处是收敛速度比较快,因为其计算比较简单,是通过不断计算距离来迭代更新的,这个距离一般是欧式距离。

\subsubsection{GSDMM算法}
GSDMM算法是一个基于Dirichlet多项式混合模型的算法,发表自清华大学信息科学与技术国家实验室的尹建华和尹建勇\cite{Yin:2014:DMM:2623330.2623715},其算法给出了一个通俗易懂的例子来理解这个算法:一位教授正在教一个电影课。在课程开始时,学生被随机分配到K个桌子。在课程开始前,学生们列出一个他们最喜欢的电影的表单。教授则反复阅读每一位学生的表单。每次叫到一位学生的时候,这位学生会被分配到至少满足以下一个条件的新桌子去:
\begin{enumerate}
	\item 新桌子的学生比当前学生所在的桌子的学生多
	\item 新桌子的学生的电影清单都是相似的
\end{enumerate}
通过不断地迭代,预期所有的学生都会被分配到“最优”的桌子去。GSDMM算法的好处是无需指定K值,该算法会自动调整K值,并且根据作者的测试结果,GSDMM的耗时比KMeans少了4/5。

\subsection{本章小结}

本章主要介绍了相关需求背景和聚类算法的发展历史,对KMeans算法和GSDMM算法进行了粗浅的初步解释,在第三章中我会更加细致地解释这两种算法。
