\newpage


\centerline{\fangsong\bf\zihao{-2}{司法语料标注系统}}
\addcontentsline{toc}{section}{摘要(关键词)}%加入目录


\vskip 1cm

\begin{center}
	\kaishu
	\hspace{2cm}计算机与软件学院计算机科学与技术专业 \quad 叶志杰
	\vspace{5bp}
	\newline
	学号:2015300091
\end{center}

\vskip 10bp

{
\kaishu	
\hspace{5bp}{\zihao{-4}\textbf{【摘要】}} 
在律师和法官的日常工作中,筛选判决书和从判决书里提取出关键纠纷占据了很大的人力成本。本文针对这一痛点,提出了自动提取法律判决书中关键纠纷的方法。训练集为一批法律判决书。首先对获取到的训练集进行中文预处理,包括中文切词、去除停用词、word2vec词向量化、构建司法领域词典等操作。然后对从原文档词向量化出来的词向量,使用Kmeans算法进行多次迭代聚类,并通过评估簇心平均距离进行快速筛选。筛选得到一批簇后,对这批簇使用svm算法进行训练,训练出来的模型已经具有能分辨法律判决书中关键纠纷的能力了。实验结果表明,此方法效果良好,能有效地从判决书中提取出关键纠纷,大幅度减少人工提取法律文书中的关键纠纷的成本。

\vskip 10bp

\hspace{5bp} {\zihao{-4}\textbf{【 关键词】}} 
法律文书; KMeans; 文本标注; word2vec; SVM; 关键因子
}